%%
%%
%% anx_latex.tex for  in cours/outils_gnu
%%
%% Made by Philippe THIERRY
%% Login   <Philippe THIERRYreseau-libre.net>
%%
%% Started on  sam. 18 sept. 2010 11:43:34 CEST Philippe THIERRY
%% Last update sam. 18 sept. 2010 11:50:40 CEST Philippe THIERRY
%%

\chapter{Exemple de documentation logicielle r�dig�e en \LaTeX}

\paragraph{}
Le script \ref{lst:latex_example} est un exemple (r�el) de documentation \LaTeX d'un projet Open-Source.

\begin{lstlisting}[caption={Documentation logicielle type r�dig�e en \LaTeX},label=lst:latex_example]
%%
%%
%% requirements.tex for libcommon in libcommon/doc/latex/includes
%%
%% Made by phil
%% Login   <philreseau-libre.net>
%%
%% Started on  mer. 09 sept. 2009 23:05:40 CEST phil
%% Last update mer. 02 juin 2010 19:58:56 CEST Philippe THIERRY
%%
\documentclass[a4paper]{book}
\usepackage{a4wide}
\usepackage[Lenny]{fncychap}
\usepackage{makeidx}
\usepackage{fancyhdr}
\usepackage[dvips]{graphicx}
\usepackage{multirow}
\usepackage{multicol}
\usepackage{float}
\usepackage{textcomp}
\usepackage{alltt}
\usepackage{times}
\usepackage{ifpdf}
\usepackage{lastpage}
%\usepackage{calc}
\usepackage{datetime}
\usepackage{chngpage}
\usepackage{tabularx}
\usepackage{listings}

\usepackage{styles/macros}

\usepackage{styles/requirement}
\usepackage[toc,page]{appendix}
\usepackage{color}
\usepackage[hundred]{vrsion}
\ifpdf
\usepackage[pdftex,
            pagebackref=true,
            colorlinks=true,
            linkcolor=blue
           ]{hyperref}
\else
\usepackage[ps2pdf,
            pagebackref=true,
            colorlinks=true,
            linkcolor=blue
           ]{hyperref}
\usepackage{pspicture}
\fi

\usepackage{titlesec}

% charts support
\usepackage{pdftricks}
\begin{psinputs}
  \usepackage{pstricks}
  \usepackage{multido}
  \usepackage{epsfig}
  \usepackage{pst-grad} % For gradients
  \usepackage{pst-plot} % For axes
\end{psinputs}

% Add a background picture to first page
\usepackage{eso-pic}
% coloration of section and subsection titiles
\definecolor{seccol}{rgb}{0.8,0,0}
\definecolor{subseccol}{rgb}{0,0,0.7}

\titleformat{\section}
{\color{seccol}\normalfont\Large\bfseries}{\thesection}{1em}{}

\titleformat{\subsection}
{\color{subseccol}\normalfont\large\bfseries}{\thesubsection}{1em}{}

\makeindex
\setcounter{tocdepth}{1}
\renewcommand{\footrulewidth}{0.4pt}

% add code support
\definecolor{colKeys}{rgb}{0,0,1}
\definecolor{colIdentifier}{rgb}{0,0,0}
\definecolor{colComments}{rgb}{0,0.5,1}
\definecolor{colString}{rgb}{0.6,0.1,0.1}
\definecolor{colList}{rgb}{0.9,0.9,0.9}

%%%% debut macro %%%%
\newenvironment{changemargin}[2]{\begin{list}{}{%
\addtolength{\leftmargin}{#1}%
\addtolength{\rightmargin}{#2}%
}\item }{\end{list}}
%%%% fin macro %%%%

\renewcommand{\chaptermark}[1]{%
\markboth{\MakeUppercase{%
\thechapter. #1}}{}}

\definecolor{mail}{rgb}{1,0,0}

% update table of contents depth
\setcounter{tocdepth}{3} % Dans la table des matieres
\setcounter{secnumdepth}{3} % Avec un numero.

\pagestyle{fancy}
\fancyhf{}

%%
%%
%% titlepage.tex for doctorat in /home/phil/Travail/Scolarité/Doctorat/doc/tex
%%
%% Made by Philippe THIERRY
%% Login   <Philippe THIERRYreseau-libre.net>
%%
%% Started on  mar. 24 nov. 2009 19:50:24 CET Philippe THIERRY
%% Last update Wed Sep  1 18:32:43 2010 Philippe THIERRY
%%

\newcommand{\thesisTitle}{
\changepage{3cm}{0cm}{0cm}{0cm}{0cm}{0cm}{-1cm}{0cm}{0cm}
\begin{center}
\huge{\textbf{ECOLE CENTRALE D'ELECTRONIQUE\\
PARIS}}\\
\vspace{1.1cm}
\large{\textsc{}}\\
\vspace{1.1cm}
Document �crit par\\
\vspace{0.5cm}
\Large{\textbf{Philippe THIERRY}}\\
\vspace{0.8cm}
dans le cadre du partenariat ECE/Thales Communication France\\
\vspace{0.5cm}
\large{\textsc{}}\\
\vspace{1.8cm}
Sujet\\
\vspace{1.0cm}
\begin{adjustwidth}{-1cm}{-1cm}
\begin{tabular}[width=19cm]{||c||}
\hline
\hline
 \\
\huge{\textsc{Les outils GNU pour la production}}\\
\huge{\textsc{Formation initiale aux outils de production et}}\\
\huge{\textsc{�tudes de cas r�els}}\\
  \\
\hline
\hline
\end{tabular}\\
\end{adjustwidth}
\vspace{2.5cm}
%\normalsize{}\\
\vspace{1cm}
%\normalsize{}\\
\vspace{0.7cm}
\begin{tabular}{lll}
%\large{M. Laurent} & \large{GEORGE} & \large{directeur de th�se} \\
\end{tabular}
\end{center}
\newpage
\changepage{-3cm}{0cm}{0cm}{0cm}{0cm}{0cm}{1cm}{0cm}{0cm}
}


\begin{document}
% Declare ref counter. This counter is used to increment the [ REF X ]
% values, for successive elements defined as a reference
\newcounter{ref}
\stepcounter{ref}

\renewcommand{\headrulewidth}{0pt}
\renewcommand{\footrulewidth}{0pt}

% Generate the titlepage background using given picture
\AddToShipoutPicture*{\BackgroundPic{share/img/libcommon-background.png}}
\libcommontitle{the requirements cookbook}{0.98.8}

\cleardoublepage
\renewcommand{\headrulewidth}{0.5pt}
\fancyhf{}
\pagestyle{fancy}
\renewcommand{\footrulewidth}{1pt}
\fancyhead[LO]{\uppercase{\leftmark}}
\fancyhead[RO]{\rightmark}
\fancyhead[LE]{\leftmark}
\fancyhead[RE]{\rightmark}
\fancyfoot[LE]{\thepage/\pageref{LastPage}}
\fancyfoot[RO]{\thepage/\pageref{LastPage}}
\fancyfoot[C]{libcommon v.0.98.8 - generated on September, 15 2010}
\pagenumbering{roman}
\tableofcontents
\cleardoublepage
\listoffigures
\cleardoublepage
\pagenumbering{arabic}

\begin{minipage}{0.8\linewidth}
\vspace{2cm}
{\Huge Abstract\\}
\vspace{2cm}
{
\paragraph{}
\it
This document describes all the input requirements of the libcommon project.\\
This document is in compliance with the DO178B-DAL C, presenting the system
requirements\footnote{see The DO-178B description, part {\it system
requirements}}. This document is the concatenation of all the specialized
design \& requirements documents.
}
\end{minipage}
\newpage

%%
%%
%% introduction.tex for  in /doctorat/ece/partenariat/cours/outils_gnu
%%
%% Made by Philippe THIERRY
%% Login   <Philippe THIERRYreseau-libre.net>
%%
%% Started on  Wed Sep  1 18:07:05 2010 Philippe THIERRY
%% Last update sam. 18 sept. 2010 11:29:39 CEST Philippe THIERRY
%%

\chapter{Introduction}

\paragraph{}
Le but de ce document est de fournir une base de connaissance suffisante permettant d'apr�hender l'architecture des
�l�ments logiciels utilis�s dans les environnements UNIX/Linux.

\paragraph{}
Le nombre de logiciels s'appuyant sur le syst�me de production d�crit ci-apr�s est tr�s important, et l'est de plus
en plus avec l'arriv�e des nouvelles plateformes logicielles embarqu�es dans la t�l�phonie ou encore la t�l�vision,
s'appuyant pour la plupart sur un socle GNU/Linux.\\
On retrouve �galement ces m�mes logiciels dans les syst�mes r�seaux, comme les {\it box}, ou encore les routeurs et
appliances\footnote{Solutions compl�tes pour r�pondre � un besoin, associant logiciels et mat�riels} (hors CISCO IOS).

\paragraph{}
Dans le m�tier des t�l�communications ou encore chez les industriels de la t�l�phonie ou les fournisseurs de
logiciels embarqu�s, les projets se basent de plus en plus fr�quemment sur le syst�me de production standardis� par
UNIX, sous sa version libre GNU (Gnu Compiler Collection). Ce document en d�crit les arcanes et met en lumi�re
les �l�ments de complexit� afin de fournir une base n�cessaire � l'�tude des grands logiciels.


% code quality
\input{code_quality.tex}
% software architecture
\input{software_architecture.tex}
% software production
\input{software_production.tex}
% code quality
\input{software_test.tex}
% software documentation
\input{software_documentation.tex}

\printindex
\end{document}
\end{lstlisting}
