%%
%%
%% makefiles.tex for  in /doctorat/ece/partenariat/cours/outils_gnu
%%
%% Made by Philippe THIERRY
%% Login   <Philippe THIERRYreseau-libre.net>
%%
%% Started on  Wed Sep  1 18:07:34 2010 Philippe THIERRY
%% Last update Fri Sep  3 18:11:53 2010 Philippe THIERRY
%%

\chapter{La commande GNU Make}

{\it Il s'agit, dans le cadre de ce chapitre, de d�crire comment on automatise la production de
gros logiciels, et quelles sont les probl�matiques qui en d�coulent}

\section{Principe des \index{Makefile}Makefiles}

\paragraph{}
Le syst�me de Makefile a �t� cr�� dans le but de simplifier la production d'un logiciel. Il
s'articule autour de fichiers de configuration, nomm�s {\texttt Makefile}, qui d�finissent les
diff�rents �l�ments des sources et le traitement qui leur est associ�.\\
Ces fichiers de configuration sont trait� par le binaire {\texttt make}, qui existe en plusieurs
versions, dont les c�l�bres BSD Make et GNU Make. Sous Linux, c'est le make de la GNU qui est
install� par d�faut. Cette version int�gre des capacit�s suppl�mentaires par rapport au make de BSD,
ce qui peut parfois rendre les Makefiles non portables si ils utilisent ces fonctionnalit�s.

\subsection{Les cibles et les d�pendances}

\subsection{La norme GNU des intitul�s de cibles}

\section{Le langage des Makefiles}

\subsection{Les variables}

\subsection{Les structures de contr�le}

\subsection{La r�cursivit�}

\subsection{Les r�gles PHONY}
\index{Makefile!PHONY}

\section{D�composer et automatiser la compilation}

\section{Les limitations}

\section{Ce qu'il ne faut pas faire}
