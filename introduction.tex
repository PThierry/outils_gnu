%%
%%
%% introduction.tex for  in /doctorat/ece/partenariat/cours/outils_gnu
%%
%% Made by Philippe THIERRY
%% Login   <Philippe THIERRYreseau-libre.net>
%%
%% Started on  Wed Sep  1 18:07:05 2010 Philippe THIERRY
%% Last update mer. 01 sept. 2010 19:52:38 CEST Philippe THIERRY
%%

\chapter{Introduction}

\paragraph{}
Le but de ce document et de fournir une base de connaissance suffisante permettant d'apréhender l'architecture des
éléments logiciels utilisé dans les environnements UNIX/Linux.

\paragraph{}
Le nombre de logiciels s'appuyant sur le système de production décrit ci-après est très important, et l'est de plus
en plus avec l'arrivée des nouvelles plateforme logicielles embarquées dans la téléphonie ou encore la télévision,
s'appuyant pour la plupart sur un socle GNU/Linux.\\
On retrouve également ces mêmes logiciels dans les systèmes réseaux, comme les {\it box}, ou encore les routeurs et
appliances (hors CISCO IOS).

\paragraph{}
Dans le métier des télécommunications ou encore chez les industriels de la téléphonie ou les fournisseurs de
logiciels embarqués, les projets se basent de plus en plus fréquement sur le système de production standardisé par
les UNIX, sous sa version libre GNU (Gnu Compiler Collection). Ce document en décrit les arcanes et met en lumière
les éléments de complexité afin de fournir une base nécessaire à l'étude des grands logiciels.
