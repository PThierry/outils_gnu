%%
%%
%% introduction.tex for  in /doctorat/ece/partenariat/cours/outils_gnu
%%
%% Made by Philippe THIERRY
%% Login   <Philippe THIERRYreseau-libre.net>
%%
%% Started on  Wed Sep  1 18:07:05 2010 Philippe THIERRY
%% Last update mer. 01 sept. 2010 19:52:38 CEST Philippe THIERRY
%%

\chapter{Introduction}

\paragraph{}
Le but de ce document et de fournir une base de connaissance suffisante permettant d'apr�hender l'architecture des
�l�ments logiciels utilis� dans les environnements UNIX/Linux.

\paragraph{}
Le nombre de logiciels s'appuyant sur le syst�me de production d�crit ci-apr�s est tr�s important, et l'est de plus
en plus avec l'arriv�e des nouvelles plateforme logicielles embarqu�es dans la t�l�phonie ou encore la t�l�vision,
s'appuyant pour la plupart sur un socle GNU/Linux.\\
On retrouve �galement ces m�mes logiciels dans les syst�mes r�seaux, comme les {\it box}, ou encore les routeurs et
appliances (hors CISCO IOS).

\paragraph{}
Dans le m�tier des t�l�communications ou encore chez les industriels de la t�l�phonie ou les fournisseurs de
logiciels embarqu�s, les projets se basent de plus en plus fr�quement sur le syst�me de production standardis� par
les UNIX, sous sa version libre GNU (Gnu Compiler Collection). Ce document en d�crit les arcanes et met en lumi�re
les �l�ments de complexit� afin de fournir une base n�cessaire � l'�tude des grands logiciels.
